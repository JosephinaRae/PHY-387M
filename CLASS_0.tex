\documentclass[10pt]{article}
\usepackage{NotesTeX} %/Path/to/package should be replaced with package location
\usepackage{lipsum}
\usepackage{tensor}
\usepackage{amsmath,amsthm,amssymb}
\usepackage{hyperref}

\newcommand{\bs}{\textbackslash}


\title{{\Huge General Relativity}\\{\Large{Class 0}}} %replace with class number
\author{Maria Jose Bustamante Rosell}

\emailAdd{majoburo@utexas.edu} %replace with your email
\begin{document}
    \maketitle
    \flushbottom
    \newpage
    \pagestyle{fancynotes}
    \part{HELLO \LaTeX\,}
	Use the uncompiled version of this document in itself as a \LaTeX\, style guide for the class you'll be responsible for.
              	\section{Class notes style package}\label{sec:class_style}
              		The package for the class notes will be NotesTex. 
    			Please read this \href{https://github.com/Adhumunt/NotesTeX/blob/master/NoTeX/NotesTeX.pdf}{document} to understand how to use the package. 
    			If you already installed \LaTeX\, it most likely already has NotesTex. 
    			Try putting \texttt{\bs usepackage\{NotesTex\}} in your preamble.  
    			Otherwise you can install it through you regular \LaTeX\, package manager or follow the instructions on the \href{https://github.com/Adhumunt/NotesTeX}{\textit{git repository}}.
    			\sn{If you are totally new to \LaTeX\, \href{https://www.latex-tutorial.com/tutorials/}{this} looks like a nice guide.}
    			\sn{If you are totally new to git, I'm sorry, you will need it. I'll talk briefly about what will be required in this document.}  
               	\section{Other useful packages}\label{sec:useful_pkg}
               		\begin{itemize}
               			\item \texttt{\bs usepackage\{tensor\}} Putting upper and lower indices in the right order has proven hard for me. If you agree, use this package. Here's a nice example of it's usage:
                				\begin{align*}
            					\partial_\lambda \tensor{\mathcal{M}}{^\mu^\nu^\lambda} &= \tensor{\delta}{^\mu_\lambda} T^{\nu \lambda}-\tensor{\delta}{^\nu_\lambda} T^{\mu \lambda}\\
                    					&= T^{\nu\mu}-T^{\mu\nu} =0
                				\end{align*}   \
               			\item \texttt{\bs usepackage\{hyperref\}} Use this at the very end of your preambles. 
    					It will make a clickable link to all your references :).
               			\item \texttt{\bs usepackage\{amsmath,amsthm,amssymb\}} The AMSmaths packages. 
    					Add math symbols and align environment (best way to write equations).
               			\item \texttt{\bs usepackage\{graphicx\}} You need this to include figures. 
    					\sn{This is kinda like a "if you are using \LaTeX\, you should use this package", this not meant to be a comprehensive list.}
               		\end{itemize}
               	\section{Best practices}\label{sec:best_prac}
               		\begin{itemize}
    				\item Label your equations, figures, tables and sections in a meaningful fashion. Like for instance, \eqref{eq:EFE}, the Einstein Field Equation:
               				\begin{align}\label{eq:EFE}
                					\tensor{G}{_\mu_\nu}=8 \pi \tensor{T}{_\mu_\nu}
               				\end{align}
    				\item Use indentation in your document to make it readable. 
    				\item Have one phrase per line.
    					\sn{
    						\LaTeX\, don't care.
    						 \LaTeX\, only care about \texttt{\bs \bs}
    						}
    			\end{itemize}
    \newpage
    \part{HELLO GIT}
    	There are plenty good videos and tutorials online on how to use git.
	Please feel free to look around, ask your peers and me or Aaron for help if you get stuck. \\
        The class will be kept in a public Github repository. 
        You can clone it, branch it, edit the part you are responsible for and then request for a merge to the main repository.
        You will be mostly evaluated on this regard. 
        You will also be evaluated on relevant comments on the rest of the class material.
        For some of you, that will be the meat of you evaluation.
        \sn{There are more students enlisted than classes in the semester.}
        If you spot something wrong in the equations, figures or text, raise an issue or push an edit with the appropriate comment.
        If you think a section is unclear or it could use any addition feel free as well to raise the issue.
\end{document}
